% Options for packages loaded elsewhere
\PassOptionsToPackage{unicode}{hyperref}
\PassOptionsToPackage{hyphens}{url}
%
\documentclass[
]{article}
\usepackage{lmodern}
\usepackage{amssymb,amsmath}
\usepackage{ifxetex,ifluatex}
\ifnum 0\ifxetex 1\fi\ifluatex 1\fi=0 % if pdftex
  \usepackage[T1]{fontenc}
  \usepackage[utf8]{inputenc}
  \usepackage{textcomp} % provide euro and other symbols
\else % if luatex or xetex
  \usepackage{unicode-math}
  \defaultfontfeatures{Scale=MatchLowercase}
  \defaultfontfeatures[\rmfamily]{Ligatures=TeX,Scale=1}
\fi
% Use upquote if available, for straight quotes in verbatim environments
\IfFileExists{upquote.sty}{\usepackage{upquote}}{}
\IfFileExists{microtype.sty}{% use microtype if available
  \usepackage[]{microtype}
  \UseMicrotypeSet[protrusion]{basicmath} % disable protrusion for tt fonts
}{}
\makeatletter
\@ifundefined{KOMAClassName}{% if non-KOMA class
  \IfFileExists{parskip.sty}{%
    \usepackage{parskip}
  }{% else
    \setlength{\parindent}{0pt}
    \setlength{\parskip}{6pt plus 2pt minus 1pt}}
}{% if KOMA class
  \KOMAoptions{parskip=half}}
\makeatother
\usepackage{xcolor}
\IfFileExists{xurl.sty}{\usepackage{xurl}}{} % add URL line breaks if available
\IfFileExists{bookmark.sty}{\usepackage{bookmark}}{\usepackage{hyperref}}
\hypersetup{
  pdftitle={8: Data Visualization Basics},
  pdfauthor={Environmental Data Analytics \textbar{} Kateri Salk},
  hidelinks,
  pdfcreator={LaTeX via pandoc}}
\urlstyle{same} % disable monospaced font for URLs
\usepackage[margin=2.54cm]{geometry}
\usepackage{color}
\usepackage{fancyvrb}
\newcommand{\VerbBar}{|}
\newcommand{\VERB}{\Verb[commandchars=\\\{\}]}
\DefineVerbatimEnvironment{Highlighting}{Verbatim}{commandchars=\\\{\}}
% Add ',fontsize=\small' for more characters per line
\usepackage{framed}
\definecolor{shadecolor}{RGB}{248,248,248}
\newenvironment{Shaded}{\begin{snugshade}}{\end{snugshade}}
\newcommand{\AlertTok}[1]{\textcolor[rgb]{0.94,0.16,0.16}{#1}}
\newcommand{\AnnotationTok}[1]{\textcolor[rgb]{0.56,0.35,0.01}{\textbf{\textit{#1}}}}
\newcommand{\AttributeTok}[1]{\textcolor[rgb]{0.77,0.63,0.00}{#1}}
\newcommand{\BaseNTok}[1]{\textcolor[rgb]{0.00,0.00,0.81}{#1}}
\newcommand{\BuiltInTok}[1]{#1}
\newcommand{\CharTok}[1]{\textcolor[rgb]{0.31,0.60,0.02}{#1}}
\newcommand{\CommentTok}[1]{\textcolor[rgb]{0.56,0.35,0.01}{\textit{#1}}}
\newcommand{\CommentVarTok}[1]{\textcolor[rgb]{0.56,0.35,0.01}{\textbf{\textit{#1}}}}
\newcommand{\ConstantTok}[1]{\textcolor[rgb]{0.00,0.00,0.00}{#1}}
\newcommand{\ControlFlowTok}[1]{\textcolor[rgb]{0.13,0.29,0.53}{\textbf{#1}}}
\newcommand{\DataTypeTok}[1]{\textcolor[rgb]{0.13,0.29,0.53}{#1}}
\newcommand{\DecValTok}[1]{\textcolor[rgb]{0.00,0.00,0.81}{#1}}
\newcommand{\DocumentationTok}[1]{\textcolor[rgb]{0.56,0.35,0.01}{\textbf{\textit{#1}}}}
\newcommand{\ErrorTok}[1]{\textcolor[rgb]{0.64,0.00,0.00}{\textbf{#1}}}
\newcommand{\ExtensionTok}[1]{#1}
\newcommand{\FloatTok}[1]{\textcolor[rgb]{0.00,0.00,0.81}{#1}}
\newcommand{\FunctionTok}[1]{\textcolor[rgb]{0.00,0.00,0.00}{#1}}
\newcommand{\ImportTok}[1]{#1}
\newcommand{\InformationTok}[1]{\textcolor[rgb]{0.56,0.35,0.01}{\textbf{\textit{#1}}}}
\newcommand{\KeywordTok}[1]{\textcolor[rgb]{0.13,0.29,0.53}{\textbf{#1}}}
\newcommand{\NormalTok}[1]{#1}
\newcommand{\OperatorTok}[1]{\textcolor[rgb]{0.81,0.36,0.00}{\textbf{#1}}}
\newcommand{\OtherTok}[1]{\textcolor[rgb]{0.56,0.35,0.01}{#1}}
\newcommand{\PreprocessorTok}[1]{\textcolor[rgb]{0.56,0.35,0.01}{\textit{#1}}}
\newcommand{\RegionMarkerTok}[1]{#1}
\newcommand{\SpecialCharTok}[1]{\textcolor[rgb]{0.00,0.00,0.00}{#1}}
\newcommand{\SpecialStringTok}[1]{\textcolor[rgb]{0.31,0.60,0.02}{#1}}
\newcommand{\StringTok}[1]{\textcolor[rgb]{0.31,0.60,0.02}{#1}}
\newcommand{\VariableTok}[1]{\textcolor[rgb]{0.00,0.00,0.00}{#1}}
\newcommand{\VerbatimStringTok}[1]{\textcolor[rgb]{0.31,0.60,0.02}{#1}}
\newcommand{\WarningTok}[1]{\textcolor[rgb]{0.56,0.35,0.01}{\textbf{\textit{#1}}}}
\usepackage{graphicx,grffile}
\makeatletter
\def\maxwidth{\ifdim\Gin@nat@width>\linewidth\linewidth\else\Gin@nat@width\fi}
\def\maxheight{\ifdim\Gin@nat@height>\textheight\textheight\else\Gin@nat@height\fi}
\makeatother
% Scale images if necessary, so that they will not overflow the page
% margins by default, and it is still possible to overwrite the defaults
% using explicit options in \includegraphics[width, height, ...]{}
\setkeys{Gin}{width=\maxwidth,height=\maxheight,keepaspectratio}
% Set default figure placement to htbp
\makeatletter
\def\fps@figure{htbp}
\makeatother
\setlength{\emergencystretch}{3em} % prevent overfull lines
\providecommand{\tightlist}{%
  \setlength{\itemsep}{0pt}\setlength{\parskip}{0pt}}
\setcounter{secnumdepth}{-\maxdimen} % remove section numbering

\title{8: Data Visualization Basics}
\author{Environmental Data Analytics \textbar{} Kateri Salk}
\date{Spring 2020}

\begin{document}
\maketitle

\hypertarget{objectives}{%
\subsection{Objectives}\label{objectives}}

\begin{enumerate}
\def\labelenumi{\arabic{enumi}.}
\tightlist
\item
  Perform simple data visualizations in the R package \texttt{ggplot}
\item
  Develop skills to adjust aesthetics and layers in graphs
\item
  Apply a decision tree framework for appropriate graphing methods
\end{enumerate}

\hypertarget{opening-discussion}{%
\subsection{Opening discussion}\label{opening-discussion}}

Effective data visualization depends on purposeful choices about graph
types. The ideal graph type depends on the type of data and the message
the visualizer desires to communicate. The best visualizations are clear
and simple. My favorite resource for data visualization is
\href{https://www.data-to-viz.com/}{Data to Viz}, which includes both a
decision tree for visualization types and explanation pages for each
type of data, including links to R resources to create them. Take a few
minutes to explore this website.

\hypertarget{set-up}{%
\subsection{Set Up}\label{set-up}}

\begin{Shaded}
\begin{Highlighting}[]
\KeywordTok{getwd}\NormalTok{()}
\end{Highlighting}
\end{Shaded}

\begin{verbatim}
## [1] "C:/Users/senam/Box Sync/My Documents/MEM classes/Duke Spring 2020/DataAnalytics/Environmental_Data_Analytics_2020"
\end{verbatim}

\begin{Shaded}
\begin{Highlighting}[]
\KeywordTok{library}\NormalTok{(tidyverse)}
\CommentTok{#install.packages("ggridges")}
\KeywordTok{library}\NormalTok{(ggridges)}
\KeywordTok{library}\NormalTok{(lubridate)}

\NormalTok{PeterPaul.chem.nutrients <-}\StringTok{ }
\StringTok{  }\KeywordTok{read.csv}\NormalTok{(}\StringTok{"./Data/Processed/NTL-LTER_Lake_Chemistry_Nutrients_PeterPaul_Processed.csv"}\NormalTok{)}
\NormalTok{PeterPaul.chem.nutrients.gathered <-}
\StringTok{  }\KeywordTok{read.csv}\NormalTok{(}\StringTok{"./Data/Processed/NTL-LTER_Lake_Nutrients_PeterPaulGathered_Processed.csv"}\NormalTok{)}
\NormalTok{EPAair <-}\StringTok{ }\KeywordTok{read.csv}\NormalTok{(}\StringTok{"./Data/Processed/EPAair_O3_PM25_NC1819_Processed.csv"}\NormalTok{)}

\NormalTok{EPAair}\OperatorTok{$}\NormalTok{Date <-}\StringTok{ }\KeywordTok{as.Date}\NormalTok{(EPAair}\OperatorTok{$}\NormalTok{Date, }\DataTypeTok{format =} \StringTok{"%Y-%m-%d"}\NormalTok{)}
\NormalTok{PeterPaul.chem.nutrients}\OperatorTok{$}\NormalTok{sampledate <-}\StringTok{ }\KeywordTok{as.Date}\NormalTok{(PeterPaul.chem.nutrients}\OperatorTok{$}\NormalTok{sampledate, }\DataTypeTok{format =} \StringTok{"%Y-%m-%d"}\NormalTok{)}
\end{Highlighting}
\end{Shaded}

\hypertarget{ggplot}{%
\subsection{ggplot}\label{ggplot}}

ggplot, called from the package \texttt{ggplot2}, is a graphing and
image generation tool in R. This package is part of tidyverse. While
base R has graphing capabilities, ggplot has the capacity for a wider
range and more sophisticated options for graphing. ggplot has only a few
rules:

\begin{itemize}
\tightlist
\item
  The first line of ggplot code always starts with \texttt{ggplot()}
\item
  A data frame must be specified within the \texttt{ggplot()} function.
  Additional datasets can be specified in subsequent layers.
\item
  Aesthetics must be specified, most commonly x and y variables but
  including others. Aesthetics can be specified in the \texttt{ggplot()}
  function or in subsequent layers.
\item
  Additional layers must be specified to fill the plot.
\end{itemize}

\hypertarget{geoms}{%
\subsubsection{Geoms}\label{geoms}}

Here are some commonly used layers for plotting in ggplot:

\begin{itemize}
\tightlist
\item
  geom\_bar
\item
  geom\_histogram
\item
  geom\_freqpoly
\item
  geom\_boxplot
\item
  geom\_violin
\item
  geom\_dotplot
\item
  geom\_density\_ridges
\item
  geom\_point
\item
  geom\_errorbar
\item
  geom\_smooth
\item
  geom\_line
\item
  geom\_area
\item
  geom\_abline (plus geom\_hline and geom\_vline)
\item
  geom\_text
\end{itemize}

\hypertarget{aesthetics}{%
\subsubsection{Aesthetics}\label{aesthetics}}

Here are some commonly used aesthetic types that can be manipulated in
ggplot:

\begin{itemize}
\tightlist
\item
  color
\item
  fill
\item
  shape
\item
  size
\item
  transparency
\end{itemize}

\hypertarget{plotting-continuous-variables-over-time-scatterplot-and-line-plot}{%
\subsubsection{Plotting continuous variables over time: Scatterplot and
Line
Plot}\label{plotting-continuous-variables-over-time-scatterplot-and-line-plot}}

\begin{Shaded}
\begin{Highlighting}[]
\CommentTok{# Scatterplot}
\KeywordTok{ggplot}\NormalTok{(EPAair, }\KeywordTok{aes}\NormalTok{(}\DataTypeTok{x =}\NormalTok{ Date, }\DataTypeTok{y =}\NormalTok{ Ozone)) }\OperatorTok{+}\StringTok{ }
\StringTok{  }\KeywordTok{geom_point}\NormalTok{()}
\end{Highlighting}
\end{Shaded}

\includegraphics{08_DataVisualization_files/figure-latex/unnamed-chunk-2-1.pdf}

\begin{Shaded}
\begin{Highlighting}[]
\NormalTok{O3plot <-}\StringTok{ }\KeywordTok{ggplot}\NormalTok{(EPAair) }\OperatorTok{+}
\StringTok{  }\KeywordTok{geom_point}\NormalTok{(}\KeywordTok{aes}\NormalTok{(}\DataTypeTok{x =}\NormalTok{ Date, }\DataTypeTok{y =}\NormalTok{ Ozone))}
\KeywordTok{print}\NormalTok{(O3plot)}
\end{Highlighting}
\end{Shaded}

\includegraphics{08_DataVisualization_files/figure-latex/unnamed-chunk-2-2.pdf}

\begin{Shaded}
\begin{Highlighting}[]
\CommentTok{# Fix this code}
\NormalTok{O3plot2 <-}\StringTok{ }\KeywordTok{ggplot}\NormalTok{(EPAair) }\OperatorTok{+}
\StringTok{  }\KeywordTok{geom_point}\NormalTok{(}\KeywordTok{aes}\NormalTok{(}\DataTypeTok{x =}\NormalTok{ Date, }\DataTypeTok{y =}\NormalTok{ Ozone), }\DataTypeTok{color =} \StringTok{"blue"}\NormalTok{)}
\KeywordTok{print}\NormalTok{(O3plot2)}
\end{Highlighting}
\end{Shaded}

\includegraphics{08_DataVisualization_files/figure-latex/unnamed-chunk-2-3.pdf}

\begin{Shaded}
\begin{Highlighting}[]
\CommentTok{# Add additional variables}
\NormalTok{PMplot <-}\StringTok{ }
\StringTok{  }\KeywordTok{ggplot}\NormalTok{(EPAair, }\KeywordTok{aes}\NormalTok{(}\DataTypeTok{x =}\NormalTok{ Month, }\DataTypeTok{y =}\NormalTok{ PM2}\FloatTok{.5}\NormalTok{, }\DataTypeTok{shape =} \KeywordTok{as.factor}\NormalTok{(Year), }\DataTypeTok{color =}\NormalTok{ Site.Name)) }\OperatorTok{+}
\StringTok{  }\KeywordTok{geom_point}\NormalTok{()}
\KeywordTok{print}\NormalTok{(PMplot)}
\end{Highlighting}
\end{Shaded}

\includegraphics{08_DataVisualization_files/figure-latex/unnamed-chunk-2-4.pdf}

\begin{Shaded}
\begin{Highlighting}[]
\CommentTok{# Separate plot with facets}
\NormalTok{PMplot.faceted <-}
\StringTok{  }\KeywordTok{ggplot}\NormalTok{(EPAair, }\KeywordTok{aes}\NormalTok{(}\DataTypeTok{x =}\NormalTok{ Month, }\DataTypeTok{y =}\NormalTok{ PM2}\FloatTok{.5}\NormalTok{, }\DataTypeTok{shape =} \KeywordTok{as.factor}\NormalTok{(Year))) }\OperatorTok{+}
\StringTok{  }\KeywordTok{geom_point}\NormalTok{() }\OperatorTok{+}
\StringTok{  }\KeywordTok{facet_wrap}\NormalTok{(}\KeywordTok{vars}\NormalTok{(Site.Name), }\DataTypeTok{nrow =} \DecValTok{3}\NormalTok{)}
\KeywordTok{print}\NormalTok{(PMplot.faceted)}
\end{Highlighting}
\end{Shaded}

\includegraphics{08_DataVisualization_files/figure-latex/unnamed-chunk-2-5.pdf}

\begin{Shaded}
\begin{Highlighting}[]
\CommentTok{# Filter dataset within plot building and facet by multiple variables}
\NormalTok{PMplot.faceted2 <-}\StringTok{ }
\StringTok{  }\KeywordTok{ggplot}\NormalTok{(}\KeywordTok{subset}\NormalTok{(EPAair, Site.Name }\OperatorTok{==}\StringTok{ "Clemmons Middle"} \OperatorTok{|}\StringTok{ }\NormalTok{Site.Name }\OperatorTok{==}\StringTok{ "Leggett"} \OperatorTok{|}
\StringTok{                  }\NormalTok{Site.Name }\OperatorTok{==}\StringTok{ "Bryson City"}\NormalTok{), }
         \KeywordTok{aes}\NormalTok{(}\DataTypeTok{x =}\NormalTok{ Month, }\DataTypeTok{y =}\NormalTok{ PM2}\FloatTok{.5}\NormalTok{)) }\OperatorTok{+}\StringTok{ }
\StringTok{  }\KeywordTok{geom_point}\NormalTok{() }\OperatorTok{+}
\StringTok{  }\KeywordTok{facet_grid}\NormalTok{(Site.Name }\OperatorTok{~}\StringTok{ }\NormalTok{Year) }
\KeywordTok{print}\NormalTok{(PMplot.faceted2)}
\end{Highlighting}
\end{Shaded}

\includegraphics{08_DataVisualization_files/figure-latex/unnamed-chunk-2-6.pdf}

\begin{Shaded}
\begin{Highlighting}[]
\CommentTok{# Plot true time series with geom_line}
\NormalTok{PMplot.line <-}\StringTok{ }
\StringTok{  }\KeywordTok{ggplot}\NormalTok{(}\KeywordTok{subset}\NormalTok{(EPAair, Site.Name }\OperatorTok{==}\StringTok{ "Leggett"}\NormalTok{), }
         \KeywordTok{aes}\NormalTok{(}\DataTypeTok{x =}\NormalTok{ Date, }\DataTypeTok{y =}\NormalTok{ PM2}\FloatTok{.5}\NormalTok{)) }\OperatorTok{+}
\StringTok{  }\KeywordTok{geom_line}\NormalTok{()}
\KeywordTok{print}\NormalTok{(PMplot.line)}
\end{Highlighting}
\end{Shaded}

\includegraphics{08_DataVisualization_files/figure-latex/unnamed-chunk-2-7.pdf}

\begin{Shaded}
\begin{Highlighting}[]
\CommentTok{# Exercise: build your own scatterplots of PeterPaul.chem.nutrients}

\CommentTok{# 1. }
\CommentTok{# Plot surface temperatures by day of  year. }
\CommentTok{# Color your points by year, and facet by lake in two rows.}
\NormalTok{PP.surface <-}\StringTok{ }\NormalTok{PeterPaul.chem.nutrients}\OperatorTok
\StringTok{  }\KeywordTok{filter}\NormalTok{(depth }\OperatorTok{==}\DecValTok{0}\NormalTok{)}

\KeywordTok{ggplot}\NormalTok{(PP.surface)}\OperatorTok{+}
\StringTok{  }\KeywordTok{geom_point}\NormalTok{(}\KeywordTok{aes}\NormalTok{(}\DataTypeTok{x=}\NormalTok{daynum, }\DataTypeTok{y =}\NormalTok{ temperature_C, }\DataTypeTok{color =}\NormalTok{ PP.surface}\OperatorTok{$}\NormalTok{year4))}\OperatorTok{+}
\StringTok{  }\KeywordTok{facet_wrap}\NormalTok{(PP.surface}\OperatorTok{$}\NormalTok{lakename, }\DataTypeTok{nrow =} \DecValTok{2}\NormalTok{)}
\end{Highlighting}
\end{Shaded}

\includegraphics{08_DataVisualization_files/figure-latex/unnamed-chunk-2-8.pdf}

\begin{Shaded}
\begin{Highlighting}[]
\CommentTok{# can't get the subset version to work here...}
\CommentTok{#ggplot(subset(PeterPaul.chem.nutrients, depth == 0))+}
\CommentTok{#  geom_point(aes(x = daynum, y = temperature_C, color = year4))+}
\CommentTok{#  facet_wrap(PeterPaul.chem.nutrients$lakename)}

\CommentTok{#2. }
\CommentTok{# Plot temperature by date. Color your points by depth.}
\CommentTok{# Change the size of your point to 0.5}
\KeywordTok{ggplot}\NormalTok{(PeterPaul.chem.nutrients, }\KeywordTok{aes}\NormalTok{(}\DataTypeTok{x=}\NormalTok{sampledate, }\DataTypeTok{y =}\NormalTok{ temperature_C, }\DataTypeTok{color =}\NormalTok{ depth))}\OperatorTok{+}
\StringTok{  }\KeywordTok{geom_point}\NormalTok{(}\DataTypeTok{size =} \FloatTok{0.5}\NormalTok{)}\OperatorTok{+}
\StringTok{  }\KeywordTok{scale_color_viridis_c}\NormalTok{(}\DataTypeTok{direction =} \DecValTok{-1}\NormalTok{)}
\end{Highlighting}
\end{Shaded}

\includegraphics{08_DataVisualization_files/figure-latex/unnamed-chunk-2-9.pdf}
\#\#\# Plotting the relationship between two continuous variables:
Scatterplot

\begin{Shaded}
\begin{Highlighting}[]
\CommentTok{# Scatterplot}
\NormalTok{lightvsDO <-}\StringTok{ }
\StringTok{  }\KeywordTok{ggplot}\NormalTok{(PeterPaul.chem.nutrients, }\KeywordTok{aes}\NormalTok{(}\DataTypeTok{x =}\NormalTok{ irradianceWater, }\DataTypeTok{y =}\NormalTok{ dissolvedOxygen)) }\OperatorTok{+}
\StringTok{  }\KeywordTok{geom_point}\NormalTok{()}
\KeywordTok{print}\NormalTok{(lightvsDO)}
\end{Highlighting}
\end{Shaded}

\includegraphics{08_DataVisualization_files/figure-latex/unnamed-chunk-3-1.pdf}

\begin{Shaded}
\begin{Highlighting}[]
\CommentTok{# Adjust axes}
\NormalTok{lightvsDOfixed <-}\StringTok{ }
\StringTok{  }\KeywordTok{ggplot}\NormalTok{(PeterPaul.chem.nutrients, }\KeywordTok{aes}\NormalTok{(}\DataTypeTok{x =}\NormalTok{ irradianceWater, }\DataTypeTok{y =}\NormalTok{ dissolvedOxygen)) }\OperatorTok{+}
\StringTok{  }\KeywordTok{geom_point}\NormalTok{() }\OperatorTok{+}
\StringTok{  }\KeywordTok{xlim}\NormalTok{(}\DecValTok{0}\NormalTok{, }\DecValTok{250}\NormalTok{) }\OperatorTok{+}
\StringTok{  }\KeywordTok{ylim}\NormalTok{(}\DecValTok{0}\NormalTok{, }\DecValTok{20}\NormalTok{)}
\KeywordTok{print}\NormalTok{(lightvsDOfixed)}
\end{Highlighting}
\end{Shaded}

\includegraphics{08_DataVisualization_files/figure-latex/unnamed-chunk-3-2.pdf}

\begin{Shaded}
\begin{Highlighting}[]
\CommentTok{# Depth in the fields of limnology and oceanography is on a reverse scale}
\NormalTok{tempvsdepth <-}\StringTok{ }
\StringTok{  }\KeywordTok{ggplot}\NormalTok{(PeterPaul.chem.nutrients, }\KeywordTok{aes}\NormalTok{(}\DataTypeTok{x =}\NormalTok{ temperature_C, }\DataTypeTok{y =}\NormalTok{ depth)) }\OperatorTok{+}
\StringTok{  }\CommentTok{#ggplot(PeterPaul.chem.nutrients, aes(x = temperature_C, y = depth, color = daynum)) +}
\StringTok{  }\KeywordTok{geom_point}\NormalTok{() }\OperatorTok{+}
\StringTok{  }\KeywordTok{scale_y_reverse}\NormalTok{()}
\KeywordTok{print}\NormalTok{(tempvsdepth)}
\end{Highlighting}
\end{Shaded}

\includegraphics{08_DataVisualization_files/figure-latex/unnamed-chunk-3-3.pdf}

\begin{Shaded}
\begin{Highlighting}[]
\NormalTok{NvsP <-}
\StringTok{  }\KeywordTok{ggplot}\NormalTok{(PeterPaul.chem.nutrients, }\KeywordTok{aes}\NormalTok{(}\DataTypeTok{x =}\NormalTok{ tp_ug, }\DataTypeTok{y =}\NormalTok{ tn_ug, }\DataTypeTok{color =}\NormalTok{ depth)) }\OperatorTok{+}
\StringTok{  }\KeywordTok{geom_point}\NormalTok{() }\OperatorTok{+}
\StringTok{  }\KeywordTok{geom_smooth}\NormalTok{(}\DataTypeTok{method =}\NormalTok{ lm) }\OperatorTok{+}
\StringTok{  }\KeywordTok{geom_abline}\NormalTok{(}\KeywordTok{aes}\NormalTok{(}\DataTypeTok{slope =} \DecValTok{16}\NormalTok{, }\DataTypeTok{intercept =} \DecValTok{0}\NormalTok{))}
\KeywordTok{print}\NormalTok{(NvsP)}
\end{Highlighting}
\end{Shaded}

\includegraphics{08_DataVisualization_files/figure-latex/unnamed-chunk-3-4.pdf}

\begin{Shaded}
\begin{Highlighting}[]
\CommentTok{# Exercise: Plot relationships between air quality measurements}

\CommentTok{# 1. }
\CommentTok{# Plot AQI values for ozone by PM2.5, colored by latitude }
\CommentTok{# Make the points 50 % transparent}
\CommentTok{# Add a line of best fit for the linear regression of these variables.}
\KeywordTok{ggplot}\NormalTok{(EPAair, }\KeywordTok{aes}\NormalTok{(}\DataTypeTok{x =}\NormalTok{ Ozone, }\DataTypeTok{y =}\NormalTok{ PM2}\FloatTok{.5}\NormalTok{, }\DataTypeTok{color =}\NormalTok{ SITE_LATITUDE))}\OperatorTok{+}
\StringTok{  }\KeywordTok{geom_point}\NormalTok{(}\DataTypeTok{alpha =} \FloatTok{0.5}\NormalTok{)}\OperatorTok{+}
\StringTok{  }\KeywordTok{scale_color_viridis_c}\NormalTok{(}\DataTypeTok{option =} \StringTok{"magma"}\NormalTok{)}\OperatorTok{+}
\StringTok{  }\KeywordTok{geom_smooth}\NormalTok{(}\DataTypeTok{method =}\NormalTok{ lm)}
\end{Highlighting}
\end{Shaded}

\includegraphics{08_DataVisualization_files/figure-latex/unnamed-chunk-3-5.pdf}

\hypertarget{plotting-continuous-vs.-categorical-variables}{%
\subsubsection{Plotting continuous vs.~categorical
variables}\label{plotting-continuous-vs.-categorical-variables}}

A traditional way to display summary statistics of continuous variables
is a bar plot with error bars. Let's explore why this might not be the
most effective way to display this type of data. Navigate to the Caveats
page on Data to Viz (\url{https://www.data-to-viz.com/caveats.html}) and
find the page that explores barplots and error bars.

What might be more effective ways to display the information? Navigate
to the boxplots page in the Caveats section to explore further.

\begin{Shaded}
\begin{Highlighting}[]
\CommentTok{# Box and whiskers plot}
\NormalTok{Nutrientplot3 <-}
\StringTok{  }\KeywordTok{ggplot}\NormalTok{(PeterPaul.chem.nutrients.gathered, }\KeywordTok{aes}\NormalTok{(}\DataTypeTok{x =}\NormalTok{ lakename, }\DataTypeTok{y =}\NormalTok{ concentration)) }\OperatorTok{+}
\StringTok{  }\KeywordTok{geom_boxplot}\NormalTok{(}\KeywordTok{aes}\NormalTok{(}\DataTypeTok{color =}\NormalTok{ nutrient)) }\CommentTok{# Why didn't we use "fill"?}
\KeywordTok{print}\NormalTok{(Nutrientplot3)}
\end{Highlighting}
\end{Shaded}

\includegraphics{08_DataVisualization_files/figure-latex/unnamed-chunk-4-1.pdf}

\begin{Shaded}
\begin{Highlighting}[]
\CommentTok{# Dot plot}
\NormalTok{Nutrientplot4 <-}
\StringTok{  }\KeywordTok{ggplot}\NormalTok{(PeterPaul.chem.nutrients.gathered, }\KeywordTok{aes}\NormalTok{(}\DataTypeTok{x =}\NormalTok{ lakename, }\DataTypeTok{y =}\NormalTok{ concentration)) }\OperatorTok{+}
\StringTok{  }\KeywordTok{geom_dotplot}\NormalTok{(}\KeywordTok{aes}\NormalTok{(}\DataTypeTok{color =}\NormalTok{ nutrient, }\DataTypeTok{fill =}\NormalTok{ nutrient), }\DataTypeTok{binaxis =} \StringTok{"y"}\NormalTok{, }\DataTypeTok{binwidth =} \DecValTok{1}\NormalTok{, }
               \DataTypeTok{stackdir =} \StringTok{"center"}\NormalTok{, }\DataTypeTok{position =} \StringTok{"dodge"}\NormalTok{, }\DataTypeTok{dotsize =} \DecValTok{2}\NormalTok{) }\CommentTok{#}
\KeywordTok{print}\NormalTok{(Nutrientplot4)}
\end{Highlighting}
\end{Shaded}

\includegraphics{08_DataVisualization_files/figure-latex/unnamed-chunk-4-2.pdf}

\begin{Shaded}
\begin{Highlighting}[]
\CommentTok{# Violin plot}
\NormalTok{Nutrientplot5 <-}
\StringTok{  }\KeywordTok{ggplot}\NormalTok{(PeterPaul.chem.nutrients.gathered, }\KeywordTok{aes}\NormalTok{(}\DataTypeTok{x =}\NormalTok{ lakename, }\DataTypeTok{y =}\NormalTok{ concentration)) }\OperatorTok{+}
\StringTok{  }\KeywordTok{geom_violin}\NormalTok{(}\KeywordTok{aes}\NormalTok{(}\DataTypeTok{color =}\NormalTok{ nutrient)) }\CommentTok{#}
\KeywordTok{print}\NormalTok{(Nutrientplot5)}
\end{Highlighting}
\end{Shaded}

\includegraphics{08_DataVisualization_files/figure-latex/unnamed-chunk-4-3.pdf}

\begin{Shaded}
\begin{Highlighting}[]
\CommentTok{# Frequency polygons}
\CommentTok{# Using a tidy dataset}
\NormalTok{Nutrientplot6 <-}
\StringTok{  }\KeywordTok{ggplot}\NormalTok{(PeterPaul.chem.nutrients) }\OperatorTok{+}
\StringTok{  }\KeywordTok{geom_freqpoly}\NormalTok{(}\KeywordTok{aes}\NormalTok{(}\DataTypeTok{x =}\NormalTok{ tn_ug), }\DataTypeTok{color =} \StringTok{"darkred"}\NormalTok{) }\OperatorTok{+}
\StringTok{  }\KeywordTok{geom_freqpoly}\NormalTok{(}\KeywordTok{aes}\NormalTok{(}\DataTypeTok{x =}\NormalTok{ tp_ug), }\DataTypeTok{color =} \StringTok{"darkblue"}\NormalTok{) }\OperatorTok{+}
\StringTok{  }\KeywordTok{geom_freqpoly}\NormalTok{(}\KeywordTok{aes}\NormalTok{(}\DataTypeTok{x =}\NormalTok{ nh34), }\DataTypeTok{color =} \StringTok{"blue"}\NormalTok{) }\OperatorTok{+}
\StringTok{  }\KeywordTok{geom_freqpoly}\NormalTok{(}\KeywordTok{aes}\NormalTok{(}\DataTypeTok{x =}\NormalTok{ no23), }\DataTypeTok{color =} \StringTok{"royalblue"}\NormalTok{) }\OperatorTok{+}
\StringTok{  }\KeywordTok{geom_freqpoly}\NormalTok{(}\KeywordTok{aes}\NormalTok{(}\DataTypeTok{x =}\NormalTok{ po4), }\DataTypeTok{color =} \StringTok{"red"}\NormalTok{) }
\KeywordTok{print}\NormalTok{(Nutrientplot6)}
\end{Highlighting}
\end{Shaded}

\begin{verbatim}
## `stat_bin()` using `bins = 30`. Pick better value with `binwidth`.
## `stat_bin()` using `bins = 30`. Pick better value with `binwidth`.
## `stat_bin()` using `bins = 30`. Pick better value with `binwidth`.
## `stat_bin()` using `bins = 30`. Pick better value with `binwidth`.
## `stat_bin()` using `bins = 30`. Pick better value with `binwidth`.
\end{verbatim}

\includegraphics{08_DataVisualization_files/figure-latex/unnamed-chunk-4-4.pdf}

\begin{Shaded}
\begin{Highlighting}[]
\CommentTok{# Using a gathered dataset}
\NormalTok{Nutrientplot7 <-}\StringTok{   }
\StringTok{  }\KeywordTok{ggplot}\NormalTok{(PeterPaul.chem.nutrients.gathered) }\OperatorTok{+}
\StringTok{  }\KeywordTok{geom_freqpoly}\NormalTok{(}\KeywordTok{aes}\NormalTok{(}\DataTypeTok{x =}\NormalTok{ concentration, }\DataTypeTok{color =}\NormalTok{ nutrient))}
\KeywordTok{print}\NormalTok{(Nutrientplot7)}
\end{Highlighting}
\end{Shaded}

\begin{verbatim}
## `stat_bin()` using `bins = 30`. Pick better value with `binwidth`.
\end{verbatim}

\includegraphics{08_DataVisualization_files/figure-latex/unnamed-chunk-4-5.pdf}

\begin{Shaded}
\begin{Highlighting}[]
\CommentTok{# Frequency polygons have the risk of becoming spaghetti plots. }
\CommentTok{# See https://www.data-to-viz.com/caveat/spaghetti.html for more info.}

\CommentTok{# Ridgeline plot}
\NormalTok{Nutrientplot6 <-}
\StringTok{  }\KeywordTok{ggplot}\NormalTok{(PeterPaul.chem.nutrients.gathered, }\KeywordTok{aes}\NormalTok{(}\DataTypeTok{y =}\NormalTok{ nutrient, }\DataTypeTok{x =}\NormalTok{ concentration)) }\OperatorTok{+}
\StringTok{  }\KeywordTok{geom_density_ridges}\NormalTok{(}\KeywordTok{aes}\NormalTok{(}\DataTypeTok{fill =}\NormalTok{ lakename), }\DataTypeTok{alpha =} \FloatTok{0.5}\NormalTok{) }
\KeywordTok{print}\NormalTok{(Nutrientplot6)}
\end{Highlighting}
\end{Shaded}

\begin{verbatim}
## Picking joint bandwidth of 10.9
\end{verbatim}

\includegraphics{08_DataVisualization_files/figure-latex/unnamed-chunk-4-6.pdf}

\begin{Shaded}
\begin{Highlighting}[]
\CommentTok{# Exercise: Plot distributions of AQI values for EPAair}

\CommentTok{# 1. }
\CommentTok{# Create several types of plots depicting PM2.5, divided by year. }
\CommentTok{# Choose which plot displays the data best and justify your choice. }

\KeywordTok{ggplot}\NormalTok{(EPAair, }\KeywordTok{aes}\NormalTok{(}\DataTypeTok{x=}\NormalTok{ EPAair}\OperatorTok{$}\NormalTok{PM2}\FloatTok{.5}\NormalTok{, }\DataTypeTok{color =} \KeywordTok{as.factor}\NormalTok{(Year)))}\OperatorTok{+}
\StringTok{  }\KeywordTok{geom_freqpoly}\NormalTok{()}
\end{Highlighting}
\end{Shaded}

\begin{verbatim}
## `stat_bin()` using `bins = 30`. Pick better value with `binwidth`.
\end{verbatim}

\includegraphics{08_DataVisualization_files/figure-latex/unnamed-chunk-4-7.pdf}

\begin{Shaded}
\begin{Highlighting}[]
\KeywordTok{ggplot}\NormalTok{(EPAair)}\OperatorTok{+}
\StringTok{  }\KeywordTok{geom_violin}\NormalTok{(}\KeywordTok{aes}\NormalTok{(}\DataTypeTok{x =} \KeywordTok{as.factor}\NormalTok{(Year), }\DataTypeTok{y =}\NormalTok{ EPAair}\OperatorTok{$}\NormalTok{PM2}\FloatTok{.5}\NormalTok{, }\DataTypeTok{fill =} \KeywordTok{as.factor}\NormalTok{(Year)))}
\end{Highlighting}
\end{Shaded}

\includegraphics{08_DataVisualization_files/figure-latex/unnamed-chunk-4-8.pdf}

\begin{Shaded}
\begin{Highlighting}[]
\KeywordTok{ggplot}\NormalTok{(EPAair)}\OperatorTok{+}
\StringTok{  }\KeywordTok{geom_bar}\NormalTok{(}\KeywordTok{aes}\NormalTok{(}\DataTypeTok{x=}\NormalTok{PM2}\FloatTok{.5}\NormalTok{, }\DataTypeTok{fill =} \KeywordTok{as.factor}\NormalTok{(Year)), }\DataTypeTok{position =} \StringTok{"dodge"}\NormalTok{, }\DataTypeTok{binwidth =} \DecValTok{4}\NormalTok{)}\OperatorTok{+}
\StringTok{  }\KeywordTok{facet_wrap}\NormalTok{(EPAair}\OperatorTok{$}\NormalTok{Year)}
\end{Highlighting}
\end{Shaded}

\includegraphics{08_DataVisualization_files/figure-latex/unnamed-chunk-4-9.pdf}

\begin{Shaded}
\begin{Highlighting}[]
\KeywordTok{ggplot}\NormalTok{(EPAair)}\OperatorTok{+}
\StringTok{  }\KeywordTok{geom_density_ridges}\NormalTok{(}\KeywordTok{aes}\NormalTok{(}\DataTypeTok{y =} \KeywordTok{as.factor}\NormalTok{(Year), }\DataTypeTok{x =}\NormalTok{ PM2}\FloatTok{.5}\NormalTok{, }\DataTypeTok{fill =} \KeywordTok{as.factor}\NormalTok{(Year)), }\DataTypeTok{alpha =} \FloatTok{0.5}\NormalTok{)}
\end{Highlighting}
\end{Shaded}

\begin{verbatim}
## Picking joint bandwidth of 2.5
\end{verbatim}

\includegraphics{08_DataVisualization_files/figure-latex/unnamed-chunk-4-10.pdf}

\end{document}
